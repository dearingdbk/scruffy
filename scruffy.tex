% Name           : scruffy.tex using theme -> hsrm-beamer-demo.sty
% Author         : Benjamin Weiss (benjamin.weiss@student.hs-rm.de)
% Version        : 0.3c
% Created on     : 05.05.2013
% Last Edited on : 16.05.2013
% Copyright      : Copyright (c) 2013 by Benjamin Weiss. All rights reserved.
% License        : This file may be distributed and/or modified under the
%                  GNU Public License.
% Description    : Beamer presentation on scruffy using hsrm theme.

\documentclass[compress]{beamer}
%--------------------------------------------------------------------------
% Common packages
%--------------------------------------------------------------------------
\usepackage[T1]{fontenc}
\usepackage{graphicx}
\usepackage{pgfpages}
\usepackage{multicol}
\usepackage{tabularx,ragged2e}
\usepackage{booktabs}
\usepackage{listings}
\lstset{ %
language=[LaTeX]TeX,
basicstyle=\normalsize\ttfamily,
keywordstyle=,
numbers=left,
numberstyle=\tiny\ttfamily,
stepnumber=1,
showspaces=false,
showstringspaces=false,
showtabs=false,
breaklines=true,
frame=tb,
framerule=0.5pt,
tabsize=4,
framexleftmargin=0.5em,
framexrightmargin=0.5em,
xleftmargin=0.5em,
xrightmargin=0.5em
}

%--------------------------------------------------------------------------
% Load theme
%--------------------------------------------------------------------------
\usetheme{hsrm}

\usepackage{dtklogos} % must be loaded after theme
\usepackage{tikz}
\usetikzlibrary{mindmap,backgrounds}

%--------------------------------------------------------------------------
% General presentation settings
%--------------------------------------------------------------------------
\title{C Style Checker}
\subtitle{Flex and Bison Implementation.}
\date{\today}
\author{Bruce Dearing}
\institute{Acadia University}

%--------------------------------------------------------------------------
% Notes settings
%--------------------------------------------------------------------------
%\setbeameroption{show notes}
\setbeameroption{show notes on second screen}

\begin{document}
%--------------------------------------------------------------------------
% Titlepage
%--------------------------------------------------------------------------

\maketitle

%\begin{frame}[plain]
%   \titlepage
%\end{frame}

%--------------------------------------------------------------------------
% Table of contents
%--------------------------------------------------------------------------
\section*{Overview}
\begin{frame}{Overview}
    % hideallsubsections - dont show the subsections of the slides.
    \tableofcontents[hideallsubsections]
\end{frame}

%--------------------------------------------------------------------------
% Content
%--------------------------------------------------------------------------

%--------------------------------------------------------------------------
% Introduction
%--------------------------------------------------------------------------
\section{Introduction}

%--------------------------------------------------------------------------
% Problem Description
%--------------------------------------------------------------------------
\subsection{Problem Description}

\begin{frame}{Problem Description}
        COMP 2103 students are required to format their programs according to a set of style guidelines.
        \note{Currently no style checker for C has been in use. \\ 
        The solution to this problem would require the
        program to have as much flexibility as possible.
        Ideally, the program would be used as the back end for a web page, so that someone could submit 
        their program and the web server would return a either confirmation that the style meets the 
        guidelines or a list of non-conforming constructs.}
\end{frame}

%--------------------------------------------------------------------------
% Solution Description
%--------------------------------------------------------------------------
\subsection{Solution Description}

\begin{frame}{Solution Description}
\begin{itemize}
    \item{Initial Approach}
        \begin{itemize}
            \item{Flex and Bison} 
        \end{itemize}
     
     \item{Second approach}
     \begin{itemize}
        \item{Flex and Bison}
        \item{GNU Indent}
     \end{itemize}

    \item{Third approach}
        \begin{itemize}
                \item{Flex and Bison}
        \item{GNU Indent}
        \item{Vim \texttt{'cindent'}}
        \end{itemize}
\end{itemize} 
\end{frame}


%--------------------------------------------------------------------------
% Solution Description Notes
%--------------------------------------------------------------------------
\note{
    \scriptsize{\begin{itemize}
    \item In order to provide a reasonable amount of functionality while still remaining flexible, 
        I initially chose to use a combination of LALR  compiler tools.
    \item Look-Ahead LR parser reading context free BNF grammar. \\
    \item Backus–Naur Form BNF 

     \item Unfortunately constructing a grammar that could also parse all the possible combinations of white space proved problematic.
     \item Therefore my second approach was to use a combination of compiler tools and a gnu open source project called indent.
     \item The indent program changes the appearance of a C program by inserting or deleting white space based on a set of supplied flags.
     \item indent can be used to format the code to the desired specifications.  

    \item As the name suggests gnu indent can handle indentation, however, it is not very flexible in implementing different levels of indentation.
    \item For indentation I use Vim's cindent, which is considerably more configurable using cinoptions.
    \end{itemize}}
}

%--------------------------------------------------------------------------
% Flex and bison
%--------------------------------------------------------------------------
\subsection{Flex and Bison}

\begin{frame}{What are Flex and Bison?}
    Flex and Bison are a set of tools originally designed for constructing compilers.
    They have proven to be very useful in building programs which handle
    structured input.
\end{frame}

%--------------------------------------------------------------------------
% Flex and Bison Notes
%--------------------------------------------------------------------------
\note{
    \begin{itemize}
        \item Flex is a fast lexical analyser generator. 
        \item It is a tool for generating programs that perform pattern-matching on text.
        \item A flex program consists of three sections, separated by \%\% lines.
        \item  The first section contains declarations and option settings. 
        \item The second section is a list of patterns and actions, and 
        \item the third section is C code that is copied to the generated scanner.
        \end{itemize}
}


%--------------------------------------------------------------------------
% Flex Structure
%--------------------------------------------------------------------------
\begin{frame}[containsverbatim]{Flex structure}
\scriptsize{\begin{verbatim}
%{     /* Declarations and optiions */
int chars = 0;
int words = 0;
int lines = 0;
%}
%%    /* Patterns and actions.  */
[a-zA-Z]+ { words++; chars += strlen(yytext); }
\n        { chars++; lines++; }
.         { chars++; }
%%
 /* C code that is copied to the generated scanner. */
main(int argc, char **argv)
{
    yylex();
    printf("%8d%8d%8d\n", lines, words, chars);
}
\end{verbatim}}
\end{frame}


%--------------------------------------------------------------------------
% Bison Structure
%--------------------------------------------------------------------------
\begin{frame}[containsverbatim]{Bison Structure}
\scriptsize{\begin{verbatim}
%{                 /* declare options */
%}                  
%token ONE TWO EOL /* declare tokens */
%start start
%%               
start
    : exp EOL     /* grammar rules */
    ;
exp
    : ONE 
    | TWO
    ;
%%                /* C code that is copied to parser */
main(int argc, char **argv)
{
    yyparse();
}
yyerror(char *s)
{
    fprintf(stderr, "error: %s\n", s);
}
\end{verbatim}}
\end{frame}

%--------------------------------------------------------------------------
% Bison Notes
%--------------------------------------------------------------------------
\note{
    \begin{itemize}
        \item Bison is a general-purpose parser generator that converts an annotated context-free grammar into a deterministic LR or generalized LR (GLR) parser.
        \item The input file for the Bison utility is a Bison grammar file The general form of a Bison grammar file is as follows:
    \end{itemize}
}

%--------------------------------------------------------------------------
% Implementation Description 
%--------------------------------------------------------------------------
\section{Implementation Description}

\begin{frame}{Style checker design}
    The C style checker is composed of four main components which perform the following
    checks:
    \begin{itemize}
        \item Comments;
        \item Indentation;
        \item Common errors, and Style;
        \item Format, and White space.
    \end{itemize}
        The components are tied together with a shell script which is accessed
    through a web based interface.
\end{frame}


%--------------------------------------------------------------------------
% Comments
%--------------------------------------------------------------------------
\subsection{Comments}
\begin{frame}[containsverbatim]{Comments}
The comment component of the style checker attempts to:
\begin{enumerate}
    \item verify that the program starts with a header comment similar to the following. and 
    \item that functions preceded by a short comment describing the function are correctly formatted.
    \end{enumerate}
\begin{multicols}{2}
\scriptsize{\begin{verbatim}
    HEADER COMMENT 
/*   
 * File:     A2P1.c
 * Author:   My Name 100123456  
 * Date:     2011/09/12  
 * Version:  1.0
 * 
 * Purpose: 
 * ...
 */ 
\end{verbatim}}
\scriptsize{\begin{verbatim}
FUNCTION COMMENT
 /*
  * Name:        my_func
  * Purpose:     ...
  * Arguments:   ...
  * Output:      ...
  * Modifies:    ...
  * Returns:     ...
  * Assumptions: ...
  * Bugs:        ...
  * Notes:       ...
  */
\end{verbatim}}
\end{multicols}
\end{frame}
%--------------------------------------------------------------------------
% Comments Design
%--------------------------------------------------------------------------
\begin{frame}[containsverbatim]{Comment Check Design}
The \textcolor{hsrmSec2}{\texttt{check\_comments}} program is a combined scanner parser.
\begin{verbatim}
%x COMMENT /* Exclusive start state.  */
%%
"/*"          { BEGIN(COMMENT); }
<COMMENT>"*/" { BEGIN(INITIAL); return(END_COMMENT);}

\end{verbatim}
\end{frame}

%--------------------------------------------------------------------------
% Comments Design NOTES 
%--------------------------------------------------------------------------

\note{
The starting comment character '/*' triggers the scanner to enter an 'exclusive' start state \texttt{<COMMENT>} to capture
comment tokens and pass those tokens off to the parser.\\
The parser then handles the tokens and verifies the comment is complete.
}

%--------------------------------------------------------------------------
% Comments Design
%--------------------------------------------------------------------------

\begin{frame}[containsverbatim]{Comment Check Design}
The parser is responsible for determining if the stream of tokens provided by the scanner conforms
to the grammar specification.
\scriptsize{\begin{verbatim}
%token  IDENTIFIER FILE_LBL AUTHOR START_COMMENT END_COMMENT VERSION DATE
%token  NAME ARGUMENTS OUTPUT MODIFIES RETURNS ASSUMPTIONS BUGS NOTES 
%token  PURPOSE LAST_VAL
%start program_body
%%
program_body
    : program_start program_comments
    ;
program_comments
    : comment_start
    | program_comments comment_start
    ;
program_start
    : START_COMMENT header_comment END_COMMENT {check_header();}
    ;
comment_start
    : START_COMMENT comment_body END_COMMENT
 ...
\end{verbatim}}
\end{frame}

%--------------------------------------------------------------------------
% Indentation
%--------------------------------------------------------------------------
\subsection{Indentation}
\begin{frame}[containsverbatim]{Indentation}
The indentation portion of the style checker attempts to verify the file has been indented correctly.
\begin{verbatim}
vim -e -s $temp_in < indent/vim_commands.scr
\end{verbatim}
\begin{exampleblock}{indent/vim\_commands.scr}
    \scriptsize{\begin{itemize}
        \item[:] \verb|set shiftwidth=4|
        \item[:] \verb|set cinoptions=e0,n0,f0,{0,}0,:2,=2,l1,b0,t0,+4,c4,C1,(0,w1|
        \item[:] \verb|normal gg=G|
        \item[:] \verb|wq|
    \end{itemize}}
\end{exampleblock}
\end{frame}

\note{

	The program begins by correctly indenting a temporary copy of the supplied file
	by passing the file into vim in execute mode and supplying a set of script commands.

	The temporary file is then compared against the original and the diff output
	is run through a lexical analyzer to report error messages
	when differences in indentation exist.  
}


%--------------------------------------------------------------------------
% Common Errors
%--------------------------------------------------------------------------
\subsection{Common Errors and Style}
\begin{frame}{Common Errors and Style}
The Common errors and style portion of the style checker consists of two components.
\begin{enumerate}
	\item  \textcolor{hsrmSec2}{\texttt{common\_errors}} program which initiates checks for:
		\begin{itemize}
			\item Common white space errors. and
			\item Code block bracket location.
		\end{itemize}
	\item  \textcolor{hsrmSec2}{\texttt{composite\_check}} program which combines the \\
			ANSI C grammar specification with a combination of additional 
			grammar productions to produce style error checks.  
\end{enumerate}

\end{frame}


%--------------------------------------------------------------------------
% Format and white space
%--------------------------------------------------------------------------
\subsection{Format and white space}
\begin{frame}{Format and white space}
\end{frame}

\section{Possible Extensions}
\begin{frame}{Possible Extensions}
\begin{description}
	\item[C Preprocessor] \hfill \\
		Construct the C style checker to first run the files through C preprocessor.
	\note
	{
		I made an initial attempt to implement Preprocessing the files before checking for errors. 
		Running through the C preprocessor proved very useful in eliminating portions of the file which are difficult to parse without it.
		It can be implemented to replaces macro definitions and join broken strings and variable together.
		and line position can be easily recalculated with the linemarkers the preprocessor inserts into the outfile.
		where I ran into problems with implementation is when the C file included local header files. 
		In order to implement the preprocessor portion a multiple file upload portion would be required.
		 
		This would also eliminate the need for a GLR parser in the Common Errors and Style portion.
    }
    \item
\end{description}
\end{frame}
\section{Summary and Conclusions}
\subsection{Demonstration}
\subsection{Required Software}
\subsection{Questions?}


\end{document}






