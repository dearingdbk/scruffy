\chapter{cinoptions}
\label{chap:cinoptions}

The `cinoptions' option sets how Vim performs indentation.  In the list below,
N represents a number of your choice (the number can be negative).  When
there is an `s' after the number, Vim multiplies the number by `shiftwidth':
1s is `shiftwidth' 2s is two times `shiftwidth', etc.  You can use a
decimal point, too: ``-0.5'' is minus half a `shiftwidth'.  The examples below
assume a `shiftwidth' of 4.

\begin{description}
    \item[$>$N] Amount added for ``normal'' indent.  Used after a line that should
          increase the indent (lines starting with ``if'', an opening brace,
          etc.) (default `shiftwidth').
\begin{verbatim}
        cino=            cino=>2        cino=>2s
          if (cond)        if (cond)      if (cond)
          {                {              {
              foo;             foo;               foo;
          }                }              }
\end{verbatim}
\clearpage
    \item[eN] Add N to the prevailing indent inside a set of braces if the
          opening brace at the End of the line (more precise: is not the
          first character in a line).  This is useful if you want a
          different indent when the `\{' is at the start of the line from
          when `\{' is at the end of the line  (default 0).
\begin{verbatim}
        cino=            cino=e2        cino=e-2
          if (cond) {      if (cond) {    if (cond) {
              foo;               foo;       foo;
          }                }              }
          else             else           else
          {                {              {
              bar;             bar;           bar;
          }                }              }
 \end{verbatim} 
    \item[nN] Add N to the prevailing indent for a statement after an ``if'',
          ``while'', etc., if it is NOT inside a set of braces.  This is
          useful if you want a different indent when there is no `\{'
          before the statement from when there is a `\{' before it
          (default 0).
\begin{verbatim}
        cino=          cino=n2          cino=n-2
          if (cond)      if (cond)        if (cond)
              foo;             foo;         foo;
          else              else          else
          {                 {             {
              bar;              bar;          bar;
          }                 }             }
\end{verbatim}
\clearpage
    \item[fN] Place the first opening brace of a function or other block in
          column N.  This applies only for an opening brace that is not
          inside other braces and is at the start of the line.  What comes
          after the brace is put relative to this brace (default 0).
\begin{verbatim}
        cino=          cino=f.5s          cino=f1s
          func()         func()             func()
          {                {                    {
              int foo;         int foo;             int foo;
\end{verbatim}
    \item[\{N] Place opening braces N characters from the prevailing indent.
          This applies only for opening braces that are inside other
          braces  (default 0).
\begin{verbatim}
        cino=            cino={.5s            cino={1s
          if (cond)        if (cond)            if (cond)
          {                  {                      {
              foo;             foo;                 foo;
 \end{verbatim}
    \item[\}N] Place closing braces N characters from the matching opening
          brace  (default 0).
\begin{verbatim}
        cino=            cino={2,}-0.5s     cino=}2
          if (cond)        if (cond)          if (cond)
          {                  {                {
              foo;             foo;               foo;
          }                }                    }
 \end{verbatim}
 \clearpage
    \item[\^{}N] Add N to the prevailing indent inside a set of braces if the
          opening brace is in column 0.  This can specify a different
          indent for whole of a function (some may like to set it to a
          negative number)  (default 0).
\begin{verbatim}
         cino=              cino=^-2          cino=^-s
           func()             func()            func()
           {                  {                 {
               if (cond)        if (cond)       if (cond)
               {                {               {
                   a = b;           a = b;          a = b;
               }                }               }
           }                  }                 }
 \end{verbatim}
    \item[LN] Controls placement of jump labels. If N is negative, the label
          will be placed at column 1. If N is non-negative, the indent of
          the label will be the prevailing indent minus N  (default -1).          
\begin{verbatim}
        cino=               cino=L2             cino=Ls
          func()              func()              func()
          {                   {                   {
              {                   {                   {
                  stmt;               stmt;               stmt;
          LABEL:                    LABEL:            LABEL:
              }                   }                   }
          }                   }                   }
 \end{verbatim}
 \clearpage
    \item[:N] Place case labels N characters from the indent of the switch()
          (default `shiftwidth').
\begin{verbatim}
        cino=              cino=:0
          switch (x)         switch(x)
          {                  {
              case 1:        case 1:
                  a = b;         a = b;
              default:       default:
          }                  }
 \end{verbatim}
 
    \item[=N] Place statements occurring after a case label N characters from
          the indent of the label  (default `shiftwidth').
\begin{verbatim}
        cino=            cino==10
          case 11:         case 11:  a = a + 1;
              a = a + 1;             b = b + 1;
 \end{verbatim}
 
    \item[\texttt{l}N] If N != 0 Vim will align with a case label instead of the
          statement after it in the same line.
\begin{verbatim}
        cino=                    cino=l1
          switch (a) {             switch (a) {
              case 1: {                case 1: {
                          break;           break;
                      }                        }
 \end{verbatim}
 \clearpage
    \item[bN] If N != 0 Vim will align a final ``break'' with the case label,
          so that case..break looks like a sort of block  (default: 0).
          When using 1, consider adding ``0=break'' to cinkeys.

\begin{verbatim}
        cino=              cino=b1
          switch (x)         switch(x)
          {                  {
              case 1:            case 1:
                  a = b;             a = b;
                  break;         break;

              default:           default:
                  a = 0;             a = 0;
                  break;         break;
          }                  }
 \end{verbatim}
 
    \item[gN] Place C++ scope declarations N characters from the indent of the
          block they are in  (default `shiftwidth').  A scope declaration
          can be ``public:'', ``protected:'' or ``private:''.
\begin{verbatim}
        cino=              cino=g0
          {                  {
              public:        public:
                  a = b;         a = b;
              private:       private:
          }                  }
\end{verbatim}

    \item[hN] Place statements occurring after a C++ scope declaration N
          characters from the indent of the label  (default
          `shiftwidth').
\begin{verbatim}
        cino=               cino=h10
           public:            public:   a = a + 1;
               a = a + 1;               b = b + 1;
 
 \end{verbatim}
    \item[pN] Parameter declarations for K\&R-style function declarations will
          be indented N characters from the margin  (default
          `shiftwidth').
\begin{verbatim}
        cino=               cino=p0         cino=p2s
          func(a, b)          func(a, b)      func(a, b)
              int a;          int a;                  int a;
              char b;         char b;                 char b;
 \end{verbatim}
    \item[tN] Indent a function return type declaration N characters from the
          margin  (default `shiftwidth').
\begin{verbatim}
        cino=            cino=t0        cino=t7
              int          int                    int
          func()           func()          func()
 \end{verbatim}
    \item[iN] Indent C++ base class declarations and constructor
          initializations, if they start in a new line (otherwise they
          are aligned at the right side of the `:')
          (default `shiftwidth').
\begin{verbatim}
        cino=                       cino=i0
          class MyClass :             class MyClass :
              public BaseClass        public BaseClass
          {}                          {}
          MyClass::MyClass() :        MyClass::MyClass() :
              BaseClass(3)            BaseClass(3)
          {}                          {}
 \end{verbatim}
 \clearpage
    \item[+N] Indent a continuation line (a line that spills onto the next)
              inside a function N additional characters  (default
              `shiftwidth').
              Outside of a function, when the previous line ended in a
              backslash, the 2 * N is used.
\begin{verbatim}
        cino=                  cino=+10
          a = b + 9 *            a = b + 9 *
              c;                           c;
 \end{verbatim}
    \item[cN] Indent comment lines after the comment opener, when there is no
          other text with which to align, N characters from the comment
          opener  (default 3).
\begin{verbatim}
        cino=              cino=c5
          /*                 /*
             text.                text.
           */                 */
 \end{verbatim}
    \item[CN] When N is non-zero, indent comment lines by the amount specified
          with the c flag above even if there is other text behind the
          comment opener  (default 0).
\begin{verbatim}
        cino=c0              cino=c0,C1
          /********            /********
            text.              text.
          ********/            ********/
 (Example uses ":set comments & comments-=s1:/* comments^=s0:/*")
\end{verbatim}
    \item[/N]    Indent comment lines N characters extra  (default 0).
\begin{verbatim}
        cino=              cino=/4
          a = b;            a = b;
          /* comment */         /* comment */
          c = d;            c = d;
 \end{verbatim}
 
    \item[(N]    When in unclosed parentheses, indent N characters from the line
          with the unclosed parentheses.  Add a `shiftwidth' for every
          unclosed parentheses.  When N is 0 or the unclosed parentheses
          is the first non-white character in its line, line up with the
          next non-white character after the unclosed parentheses
          (default `shiftwidth' * 2).
\begin{verbatim}
        cino=                     cino=(0
          if (c1 && (c2 ||          if (c1 && (c2 ||
                      c3))                     c3))
              foo;                      foo;
          if (c1 &&                 if (c1 &&
                  (c2 || c3))           (c2 || c3))
             {                         {
 \end{verbatim}
    \item[uN] Same as (N, but for one level deeper  (default `shiftwidth').
\begin{verbatim}
        cino=                      cino=u2
          if (c123456789             if (c123456789
                  && (c22345                 && (c22345
                      || c3))                  || c3))
 \end{verbatim}
    \item[UN] When N is non-zero, do not ignore the indenting specified by
          `(' or `u' in case that the unclosed parentheses is the first
          non-white character in its line (default 0).
\begin{verbatim}
        cino= or cino=(s      cino=(s,U1
          c = c1 &&             c = c1 &&
              (                     (
               c2 ||                    c2 ||
               c3                       c3
              ) && c4;              ) && c4;
 \end{verbatim}
  \clearpage
    \item[wN] When in unclosed parentheses and N is non-zero and either
          using ``(0'' or ``u0'', respectively, or using ``U0'' and the unclosed
          parentheses is the first non-white character in its line, line
          up with the character immediately after the unclosed parentheses
          rather than the first non-white character (default 0).
\begin{verbatim}
        cino=(0                  cino=(0,w1
          if (   c1                if (   c1
                 && (   c2             && (   c2
                        || c3))            || c3))
              foo;                     foo;
 \end{verbatim}
    \item[WN] When in unclosed parentheses and N is non-zero and either
          using ``(0'' or ``u0'', respectively and the unclosed parentheses is
          the last non-white character in its line and it is not the
          closing parentheses, indent the following line N characters
          relative to the outer context (i.e. start of the line or the
          next unclosed parentheses)  (default: 0).
\begin{verbatim}
        cino=(0                      cino=(0,W4
          a_long_line(                 a_long_line(
                      argument,            argument,
                      argument);           argument);
          a_short_line(argument,    a_short_line(argument,
                       argument);                argument);
 \end{verbatim}
 \clearpage
    \item[mN] When N is non-zero, line up a line starting with a closing
          parentheses with the first character of the line with the
          matching opening parentheses  (default 0).
\begin{verbatim}
        cino=(s              cino=(s,m1
          c = c1 && (          c = c1 && (
              c2 ||                c2 ||
              c3                   c3
              ) && c4;         ) && c4;
          if (                 if (
              c1 && c2             c1 && c2
             )                 )
              foo;                 foo;
 \end{verbatim}
    \item[MN] When N is non-zero, line up a line starting with a closing
          parentheses with the first character of the previous line
          (default 0).
\begin{verbatim}
        cino=                   cino=M1
          if (cond1 &&            if (cond1 &&
                 cond2                   cond2
             )                           )
 \end{verbatim}
 
    \item[)N] Vim searches for unclosed parentheses at most N lines away.
          This limits the time needed to search for parentheses  (default
          20 lines).

    \item[*N]    Vim searches for unclosed comments at most N lines away.  This
          limits the time needed to search for the start of a comment
          (default 70 lines).

    \item[\#N]    When N is non-zero recognize shell/Perl comments, starting with
          `\#'.  Default N is zero: don't recognizes `\#' comments.  Note
          that lines starting with `\#' will still be seen as preprocessor
          lines.

\end{description}
\clearpage
The defaults, spelled out in full, are:
\begin{verbatim}
    cinoptions=>s,e0,n0,f0,{0,}0,^0,L-1,:s,=s,l0,b0,gs,hs,ps,ts,is,+s,
            c3,C0,/0,(2s,us,U0,w0,W0,m0,j0,J0,)20,*70,#0
\end{verbatim}

The options used to obtain correct indentation for `\emph{scuffy}' are:
\begin{verbatim}
    cinoptions=e0,n0,f0,{0,}0,:2,=2,l1,b0,t0,+4,c4,C1,(0,w1
\end{verbatim}

Vim puts a line in column 1 if:
\begin{itemize}
    \item It starts with `\#' (preprocessor directives), if cinkeys contains `\#'.
    \item It starts with a label (a keyword followed by `:', other than ``case'' and
  ``default'') and `cinoptions' does not contain an `L' entry with a positive
  value.
  \item Any combination of indentations causes the line to have less than 0
  indentation.
\end{itemize}
\end{description} 
