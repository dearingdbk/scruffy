\chapter{Possible Extensions}
\label{chap:EXTENSION}

\section{Preprocessor}

A multipart extension idea would be to extend the web based uploader to handle 
multiple file uploads. This addition would allow the C style checker to be 
extended to have it's files run through the C preprocessor before parsing 
starts. This implementation will have the added benefit of no longer requiring
a GLR parser for composite check. I made an initial attempt to implement 
preprocessing of the files before checking for errors. Running through the C 
preprocessor proved very useful in eliminating portions of the file which are 
difficult to parse without it. It can be implemented to replaces macro 
definitions and join broken strings and variables together. 
Original line position can be easily recalculated with the linemarkers the 
preprocessor inserts into the outfile. Where I ran into problems with 
implementation is when the C file included local header files. In order to 
implement the preprocessor portion a multiple file upload portion would be 
required.

\section{Additional Productions}

Continue constructing grammar productions to produce checks for additional 
style errors as they are identified.

\section{Replace Indent and Vim}

A program was identified that could potentially replace indent. The program 
\emph{Astyle} on initial inspection appears to have the flexibility to be able 
to replace indent and possibly Vim's indentation capabilities as well.

\section{Javascript}

It may be possible to construct a more dynamic and user friendly web 
interface using jQuery or other javascript libraries that would provide the 
required functionality. Currently the line numbers displayed on the code view
portion are loaded by counting the number of lines in the file and then running
a for loop to print the numbers off. This means if the student inserts a 
newline into the text of their code file the line numbers do not dynamically 
adjust to match. The code view portion provides the ability to save the 
document after it has been edited, however, it is being saved as either a new 
file or opened in a text editor. It would be a beneficial feature if that when
you pressed save it could save the changes to the original uploaded document.
This may be asking to much of the system and the possibilities of accidentally
deleting your original file might outweigh the benefits.


\section{Comments Extension}

The current version of the check comments module uses regular expression to 
parse header and function comments. The tokens the comments scanner provides 
should be separated out and the grammar file extended to parse the data more 
accurately. The comments portion could also be merged with the composite check
module as the composite check is currently ignoring comments.
