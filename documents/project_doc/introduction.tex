
%-----------------------------------------------------------------------------
% Chapter: Introduction
%-----------------------------------------------------------------------------

\chapter{Introduction}
\label{chap:INTRO}

Despite the seemingly endless supply of code beautifiers available for use, 
I was unable to locate an open source solution that provided a method of 
identifying errors to the user. Code beautifiers or formatters are typically 
designed to only re-format the code that they are supplied with. The initial 
problem with having students use a code beautifier to format their code, is 
that by design code beautifiers are vastly configurable and could not provide 
us with a method of enforcing a strict coding standard. Therefore the only 
reasonable solution to the unavailability of a C style checker would be to 
construct one based on existing models for other languages. The model which my 
program was blueprinted from is checkstyle. Checkstyle is a development tool 
designed to help programmers write Java code that adheres to a coding standard 
by identifying errors in their code and reporting them to the user.

The main objective of the project was to provide a C style checker similar to 
checkstyle that students could use to verify their code before submission. 
The program will check student code against a set of available checks. 
The program will then generate a report of the errors for the student to 
action. The feedback provided by the program to the user will allow the student
to learn from their mistakes and provide a method of enforcing a uniform coding
standard.

In the remainder of this report I will discuss the tools that I used to 
construct a C style checker program, my implementation, and any extensions 
that would prove beneficial.
