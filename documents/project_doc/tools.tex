\chapter{Description of tools}
\label{chap:TOOLS}


%-----------------------------------------------------------------
% Section: Flex
%-----------------------------------------------------------------
\section{Flex}

The {\Large\textbf{F}}ast {\Large\textbf{Lex}}ical analyzer generator is a tool
for generating scanners. A \emph{scanner} is a program which recognizes
regular expression patterns in text and then performs an action.
The operation of Flex can be described as in \citep*{FLEX}:
 \begin{quote}
``The Flex program reads user-specified input files for a description 
of a scanner to generate. The description is in the form of 
pairs of regular expressions and C code, called rules. Flex generates a C 
source file named, `lex.yy.c', which defines the function yylex(). The file 
`lex.yy.c' can be compiled and linked to produce an executable. When the 
executable is run, it analyzes its input for occurrences of text matching the 
regular expressions for each rule. Whenever it finds a match, it executes the 
corresponding C code.''
\end{quote}

\subsection{Flex Input File}

A Flex scanner description file consist of three main parts, which are 
separated from each other by \emph{\%\%} lines.

\subsubsection{Options and Declarations}

\begingroup
    \fontsize{8pt}{8pt}\selectfont
\begin{verbatim}
\%e 1019  /* number of parsed tree nodes */
\%p 2807  /* number of positions */
\%n 371   /* number of states /*
\%k 284   /* number of packed character classes */
\%a 1213  /* number of transitions */
\%o 1117  /* size of output array */
O [0-7]
D [0-9]
NZ [1-9]
L [a-zA-Z_]
A [a-zA-Z_0-9]
H [a-fA-F0-9]
ES (\\(['"\?\\abfnrtv]|[0-7]{1,3}|x[a-fA-F0-9]+))
WS [ \t\v\f]
\%option yylineno

/* START Exclusive Start State definitions. */
\%x COMMENT
/* END Exclusive Start State definitions. */

/* START Inclusive Start State definitions. */
\%s TYPE\_DEF
/* END Inclusive Start State definitions. */

%{
#define YY_USER_ACTION yylloc.first_line = yylloc.last_line = yylineno;
#include "check_comments.tab.h"

int yycolumn = 1;

void add_entry(int line_num, int flag, char* str);

%}

\end{verbatim}
\endgroup

\noindent This is where we can define the options that we want Flex to include 
in the scanner. Flex offers several hundred options; most can be written as\\
\%option name \\
This section is also where we would define any inclusive or exclusive start
states that we might require in the scanner, as well as any regular expression
simplifications. We can also include any 
header files or macro definitions inside \emph{\%\{ \}\%} blocks.


\subsubsection{Token Action Section} 
\noindent The second section is where we construct our list of regular expression and 
action pairs.

\begingroup

\begin{verbatim}
"/*"                           { BEGIN(COMMENT); 
                                 return(START_COMMENT); 
                               }
"//".*                         { /* consume //-comment */ }
<COMMENT>"*/"                  { BEGIN(INITIAL); return(END_COMMENT);}
<COMMENT>\n                    { }
<COMMENT>^{BG}{F}{WS}+{FN}     { return(FILE_LBL);}
<COMMENT>^{BG}{A}{TH}{WS}+{BN} { return(AUTHOR); }
<COMMENT>^{BG}{V}              { return(VERSION); }
<COMMENT>^{BG}{D}{WS}+{DT}     { return(DATE); }
<COMMENT>^{BG}{P}              { return(PURPOSE);}
<COMMENT>^{BG}{N}{ME}          { return(NAME); }
<COMMENT>^{BG}{A}{RG}          { return(ARGUMENTS); }
<COMMENT>^{BG}{O}              { return(OUTPUT); }
<COMMENT>^{BG}{M}              { return(MODIFIES); }
<COMMENT>^{BG}{R}              { return(RETURNS); }
<COMMENT>^{BG}{A}{SS}          { return(ASSUMPTIONS); }
<COMMENT>^{BG}{B}              { return(BUGS); }
<COMMENT>^{BG}{N}{OT}          { return(NOTES); }
<COMMENT>[^*\n]|.              { }
{WS}                           { /* white space separates tokens */ }
.                              { /* discard bad characters */ }
\end{verbatim}
\endgroup

\noindent We have to be careful in the construction and ordering of our regular 
expressions in this portion, as Flex uses a greedy match, and any regular 
expressions below a larger match will never receive token matches. Fortunately 
Flex is intelligent enough to let us know when we make a mistake like this.\\
\verb|  check_comments.l:142: warning, rule cannot be matched|

\subsubsection{C Code Section}
\noindent The third section is C code that is copied to the generated scanner.
In the declaration section, code inside of \emph{\%\{} and \emph{\%\}} is copied near the 
beginning of the generated C source file. 
\begingroup

\begin{verbatim}

void
add_entry(int line_num, int flag, char* str)
{
    char *rtnstr = fix_string(str);
}


\end{verbatim}
\endgroup
\noindent In this section we can place small subroutines that may be required
as part of the action on a regular expression match.\\

\subsubsection{Flex Summary}
By utilizing Flex's ability to match regular expressions paired with exclusive 
and inclusive start states we can easily construct a highly configurable 
scanner for use in a C style checker. 

%-----------------------------------------------------------------
% Section: Bison
%-----------------------------------------------------------------
\section{Bison}
GNU Bison parser generator is a tool for generating LR or GLR parsers.
A \emph{parser} is a program which takes input provided to it from a scanner
and takes that input and constructs an abstract syntax tree and checks the 
syntax based on supplied rules. The operation of Flex can be described as in 
\citep*{BISON}:
\begin{quote}
``Bison is the gnu general-purpose parser generator that converts an annotated 
context-free grammar into a deterministic LR or generalized LR (GLR) parser.'' 
\end{quote}


\subsection{Bison Input File}

\subsubsection{Options and Declarations}

Similar to a Flex scanner description file a Bison grammar rule file consists 
of three main parts, which are separated from each other by \emph{\%\%} lines.
\begingroup
\begin{verbatim}
%locations
%glr-parser
%expect 2
%expect-rr 0

%{

/* START Include files. */
#include <stdio.h>
#include <string.h>
/* END Include files. */

/* START Program variables */
extern char *yytext;
/* End Program variables */


/* START Definitions. */
#define YYERROR_VERBOSE
#define MAX_TRYS 1000
/* End Definitions. */

/* START Function Definitions */
void yyerror(const char *str);
/* START Function Definitions */
%}

%token BIDENTIFIER IDENTIFIER I_CONSTANT F_CONSTANT STRING_LITERAL
%start program_body

\end{verbatim}
\endgroup
\noindent This is where we can define the additional options, tokens and 
starting token that we want Bison to include in the parser. This section is 
also where we would include any header files or macro definitions inside 
\emph{\%\{ \}\%} blocks.


\subsubsection{Rule Declarations} 
\noindent The second section contains simplified BNF rules.
\begingroup
\begin{verbatim}
generic_selection
    : GENERIC '(' assignment_expression ',' generic_assoc_list ')'
    ;

generic_assoc_list
    : generic_association
    | generic_assoc_list ',' generic_association
    ;

generic_association
    : type_name ':' assignment_expression
    | DEFAULT ':' assignment_expression { check_list(); }
    ;
\end{verbatim}
\endgroup
\noindent This is where we supply the rules for productions and reductions of 
program input. C code can also be supplied inside the structure of a rule 
inside curly brackets.

\subsubsection{C Code Section}
\noindent The third section is C code that is copied to the generated parser.
In the declaration section, code inside of \emph{\%\{} and \emph{\%\}} is 
copied near the beginning of the generated C source file. 
\begingroup
\begin{verbatim}

void
add_entry(int line_num, int flag, char* str)
{
    char *rtnstr = fix_string(str);
}



\end{verbatim}
\endgroup
\noindent In this section we can place small subroutines that may be required
as part of the action on a regular expression match.\\

\subsubsection{Bison Summary}
A Bison grammar input file can be carefully constructed to allow parsing of 
incredibly complex languages. The combination of Flex and Bison generator tools
provide the required functionality to implement a C style checker. 


%-----------------------------------------------------------------
% Section: Indent
%-----------------------------------------------------------------
\section{Indent}

GNU \emph{indent} is a code formatter which modifies C code by inserting or deleting
white space, based on a set of flags supplied by the user on the command line.
For a reference on available flags see \citep*{INDENT}.
Constructing a grammar which parses a complex language like C would quickly 
become too complex if we required it to handle all the possible combinations of
white space tokens. Therefore \emph{indent} is employed to handle the required 
white space changes within the C style checker. 

\subsubsection{Indent Summary}
The \emph{indent} portion of the program originally proved very useful in 
formatting code to the required settings, however, as the project progressed 
\emph{indent} was more of a hindrance than a help. The \emph{indent} program 
proved to be inflexible in it's flag selection. When choosing a particular 
style if two forms of styling were acceptable, there was no clear way to allow
both. It was impossible to turn off code modification for a particular style, 
selection of one style was enforced.  As well \emph{indent} modified code without the 
option to select a particular style. For example \verb|char* string| or 
\verb|char * string| would become \verb|char *string| with no ability to 
select a style. In future designs I hope to improve \emph{indent'}s source 
code or replace it with an alternate solution.


%-----------------------------------------------------------------
% Section: Vim
%-----------------------------------------------------------------


\section{Vim}

As the name suggests gnu \emph{indent} can handle indentation, however, it is not 
very flexible in implementing different levels of indentation.
\emph{Vim} or Vi IMproved is a non-toy editor which has numerous useful features.
One of them being \emph{cindent}, with it's ability to flexibly re-indent code 
according to a set of flags supplied to it through a variable 
\emph{cinoptions}. See \autoref{chap:cinoptions} for details.

\subsubsection{Vim Summary}
\emph{Vim} has proven to be very useful strictly as a flexible indentation 
component of the style checker, however, during the course of looking for a 
replacement program for \emph{indent}, a possible solution has been found that 
could effectively replace \emph{Vim} and \emph{indent}. Currently 
\texttt{Astyle} is under review for addition to the program as a replacement.


