\chapter{Summary and Conclusions}
\label{chap:SUMMARY}


The main objective of the project was to provide a C style checker similar to 
checkstyle that students could use to verify their code before submission. The 
style checker has been in testing with Dr.~Diamond's COMP 2103 students since 
September 29\textsuperscript{th} 2013. Without the ability to log whether the 
program is being utilized it is currently undetermined if the project is 
performing as expected, however, TA's have reported fewer style errors 
recently. A long term goal for this project would be to continue logging the 
results of its use to see if the program aids in reducing the number of 
repetitive style errors of students. An additional objective of the project was
to produce a flexible and extendable program, with the implementation of additional 
grammar productions as additional style errors are identified the program can 
be extended to incorporate the additional style errors.

In conclusion the C style checker will provide a method to enforce a uniform 
coding style throughout the COMP 2103 students which was previously unavailable.
The C style checker has the potential to produce a reduction in the number of 
common style errors present in COMP 2103 student code. The additional ability 
to receive immediate corrective feedback may help to increase retention of the 
coding standards guide line.
