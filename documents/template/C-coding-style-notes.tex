\chapter{C coding style notes}
\label{chap:C-coding-style-notes}
This document gives the basic information about acceptable coding
style for C programs in COMP 2103.
Written by Jim Diamond, Acadia University
Last updated October 2, 2013.
\begin{description}
\item[Header comment] \hfill

    Your program should start with a header comment similar to the following\:
\begin{verbatim}
    /*
     * File:     A2P1.c
     * Author:   Jim Diamond   000123456
     * Date:     2011/09/12
     * Version:  1.0
     *
     * Purpose:
     * ...
     */
\end{verbatim}
    Note the following points:
    \begin{enumerate}
    \item[(a)] The '/*' and '*/' are on lines by themselves.
    \item[(b)] There is (at least) one space between other '*'s and the
        following word(s), if any.
    \item[(c)] You need ALL of the indicated items above, although, of
        course, you will use your own name and ID, and you will use
        appropriate values for the file name, the date, and the version.
    \item[(d)] The purpose should give a reasonable concise description of
        the program.  Not as much detail as the assignment question
        itself, but enough detail so that someone who knows nothing
        about the assignment or your program could get a very good idea 
        of what the program's purpose is by reading the comment.
    \item[(e)]The "File:" name is the actual name of the file.  The example
        above is a C program file (as evidenced by the ".c");
        a shell program would not normally have any file name extension.
\end{enumerate}

\item[Line length] \hfill

    Lines must not be longer than 79 characters.
    Should you need to break a "logical" line up into two or more
    "physical" lines in order to keep the line length from being too
    long,
\begin{enumerate}
    \item[(a)] choose reasonable places in the line (from a syntactic point
        of view) to break the line, and
    \item[(b)] indent the continuation lines an amount which makes it clear
        to the reader where the continued text "fits in".
    \end{enumerate}
    For example:
\begin{verbatim}
        if (some_long_variable_name != some_long_function_name(value)
            && short_var_nm == something_else)
        {
            ...
        }
\end{verbatim}
    Note that the "\&\&" is at the beginning of the continuation line,
    rather than at the end of the previous line, making it very
    obvious that that is a continuation line.
\clearpage
    For functions with long argument lists, a common style (which I
    expect you to use) is to have the arguments on the second and
    subsequent lines lined up with the first argument on the first line.
\begin{verbatim}
    printf("The number of snarfles is %d, %d, %d and sometimes %d\n",
           tuesday_snarfles, weekly_special_snarfles + 43,
           speciality_snarfles, other_snarfles);
\end{verbatim}
\item[Variable naming conventions] \hfill

    If you use any information that is a so-called "magic number", you
    should \#define a macro whose name is composed of upper case
    letters and, optionally, the underscore character ('\_'). 
    For example,
\begin{verbatim}
        #define   DAYS_IN_WEEK   7
\end{verbatim}
    "Ordinary" variables should have names composed of lower case
    letters and, optionally, digits and underscore.  If you like
    variables where mixed case is used for readability, such as
      \verb|numberOfEmployees|
    that is acceptable, but consider whether something like
      \verb|number_of_employees|
    is more readable or not.

\item[Variable declaration conventions] \hfill

    You may declare multiple variables on one line:
    int i, j, k;
    But if you initialize any variables in a declaration statement,
    the initialized variable must be the only declared variable in
    that statement.  For example:
\begin{verbatim}    
    BAD   int i, j = 0, k;

    GOOD  int i, k;
          int j = 0;

\end{verbatim}
\clearpage
\item[Indentation] \hfill

    Your code MUST be properly indented.  I suggest (extremely strongly)
    that you use 4 spaces per indentation level.  Further, 8 is really
    too many because with only a few levels of indentation, you may
    have difficulty writing lines that don't exceed 79 characters in
    length.  Regardless of what you use, be consistent!  Here is a
    sample of properly indented code:
\begin{verbatim}
    if (answer == 3  ||  answer < 0)
    {
        for (i = 0; i < n; i++)
        {
            switch (i % 2)
            {
              case 0:
                printf("%d is even\n", i);
                break;
                
              case 1:
                printf("%d is odd\n", i);
                break;
                
              default:
                printf("%d is extremely odd\n", i);
            }
        }
    }
\end{verbatim}
    Note that the "case" labels are indented by only two spaces.
    This allows them to stick out noticeably without causing the
    code to get indented too far.

\clearpage
\item[Tab characters] \hfill

    You may use tab characters to indent your code, but only if each
    tab character is equivalent to 8 characters.  (More specifically,
    the character following a tab character is in column number 8n+1
    for some n.)

    This means that your indentation could consist of a mix of tab
    characters and spaces.  But repeat after me: "Tab stops are every
    8 spaces, as God intended."


\item[Syntactic elements and line structure] \hfill

    At Acadia we expect C and Java programs to have the open brace
    ('\{') of a code block or function/method definition on a line by
    itself.  The close brace must also be on its own line, except in
    these cases
    \begin{enumerate}
    \item[(a)] do-while statements:
\begin{verbatim}
    do
    {
        ...
    } while (a < b);
\end{verbatim}
    \item[(b)] typedef'ing a struct:
\begin{verbatim}
    typedef struct
    {
        ...
    } mytype_T;
\end{verbatim}
    \item[(c)] combining a struct definition with a variable declaration:
\begin{verbatim}
    struct mystruct
    {
        ...
    } ms1, ms2;
\end{verbatim}
\end{enumerate}
    Finally, these rules don't apply to braces when used to initialize
    arrays, such as in
    \verb|int time_units[] = {60, 60, 24, 365};|

\clearpage
\item[Comments] \hfill

    Aside from the header comment above, you should use comments when
    a few words of wisdom would enlighten a reasonably competent
    programmer who is reading your code.  If you have done something
    particularly clever or devious, explain it.  Also, if you have
    done something which makes use of some fact(s) that even a
    reasonably competent programmer wouldn't know, then explain it.
    But never, ever use comments like this:
 \begin{verbatim}       /* set i to 0 */
        i = 0;
\end{verbatim}
    If you have multi-line comments, make sure that all the '*'s of
    the same comment line up vertically.

    Note: gcc (in C90 mode) allows you to use // comments, but this is
    not portable to all C90 compilers, so if you do this, your program
    won't be as portable as if you use /* ... */ comments.
    Since we are using the C99 standard, you are welcome to use // for
    comments, but be aware that if you use a non-gcc C90 compiler, you
    may have to edit your code.


\item[White space]\hfill

    The keywords "if", "while", "for" and so on must be followed by a
    space, before the open parenthesis.  See the sample code above in
    "(4) Indentation" for examples of this style.

    Use extra white space when you think it will make a statement
    easier to read.  For example, if you compare
\begin{verbatim}
        if (answer==3||answer<0)
\end{verbatim}
    to
\begin{verbatim}
        if (answer == 3 || answer < 0)
\end{verbatim}    
    or even
\begin{verbatim}
        if (answer == 3  ||  answer < 0)
 \end{verbatim}
    you might agree the white space makes it easier to separate the
    tokens.

    Use blank lines between (relatively) unrelated blocks of code.
    For example, if some function (main() or other) has to do
    three things, but each one is more or less independent of the
    others, then separate those blocks of code with a blank line.
    For example:
\begin{verbatim}
        some code to do the first thing
        in a sequence of things to do
        for some program or function.

        some code which does some stuff that is, by itself,
        more or less independent of the stuff above.

        yet another block of code which does
        stuff not directly related to either
        of the above two blocks of code, except, of course,
        that all three blocks are used inside the same program.
\end{verbatim}
    Having said that:
\begin{itemize}
            \item[(a)] don't use white space between a function name and the \verb|'('|: \hfill
            
            \verb|z = sqrt(y);  // Not   z = sqrt (y);|
            \item[(b)] don't use white space after the \verb|')'| when casting: \hfill
            
            \verb|i = (int)f;   // Not   i = (int) f;|
            \end{itemize}
\clearpage
\item[Functions] \hfill

    Using functions allows you to manage the complexity of your
    program, making it easier to write in the first place, easier to
    test and debug, and quicker to finish.

    The body of a function should be indented.  For example:
\begin{verbatim}
    int
    my_func(...args...)
    {
        some code;
        some code;
        some code;
        ...
    }
\end{verbatim}
    I like putting the return type of a function on the previous line
    like that, because I can then do a regexp search for
        \verb|^my_func|
    which will take me to the definition of the function, not some
    place where it is used.  But putting the type of the function on
    the same line is fine if you want.
    \clearpage
\item[Function comments] \hfill

    A function should be preceded by a short comment describing the
    purpose, the inputs (if any), the outputs (if any), side effects
    (if any), and any other information which would allow a reader to
    quickly understand the purpose and use of the function.  Here is a
    comment which contains all of the items that are normally relevant.
    If you have something that you think should be said which isn't
    in one of these categories, go ahead and add it.
\begin{verbatim}
  /*
   * Name:        my_func
   * Purpose:     add up the sizes, in bytes, of all the files given
   *              as arguments
   * Arguments:   zero or more filenames
   * Output:      sum of the sizes of those files
   * Modifies:    writes sum of sizes to stdout
   * Returns:     0 on success, 1 if there were files whose sizes 
   *              could not be determined
   * Assumptions: none
   * Bugs:        none
   * Notes:       any interesting comments about this function,
   *              whether they deal with the arguments, the 
   *              algorithm employed, or the format of the output.
   */
\end{verbatim}
\end{description}
