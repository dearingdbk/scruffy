\chapter{User Manual}
\label{chap:USER}


\section{usage}

\subsection{Command line}

The command line usage of the components is simplified through the use of a 
shell script included with the program.
usage: \texttt{sh scruffy.sh $<$filename$>$} where \texttt{$<$filename$>$} is the name
of the file that should be inspected for errors.
\vskip 1em


{\large\textbf{CAUTION!}} \hfill \\
scruffy deletes any supplied file once the file has been check for style errors.
Always backup files before testing.


\subsection{Web based}

The web based version is as simple as identifying a file to upload and clicking
submit.


They C style checker performs best with C files, scanning files other than C 
code files is undocumented and unadvised.
