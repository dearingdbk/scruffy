\chapter{Implementation Description}
\label{chap:IMPLEMENT}

The core components of the C style checker are composed of four translator 
based programs, which with the help of indent and Vim provide the 
ability to perform a number of checks on submitted files for a list of 
available checks see \autoref{chap:Available Checks}. The reasoning behind 
using translator based programs is based in the fact that they would be highly 
configurable and each module could be easily updated or extended by modifying 
the scanner description and parser grammar if required. 

\section{Comments}

For code comments the C style checker is required to verify code comments 
conform to the style guide as well as verify that files minimally contain a 
header comment identifying all the required data for the file, for more 
information on acceptable comment style see 
\autoref{chap:C-coding-style-notes}. The format and placement of comments is
currently handled by indent. As in all parts related to indent, the file is 
passed through indent and then compared against the original file to display 
required changes to the user. 

\subsection{indent's modification of comments}

indent formats both C and C++ comments. \emph{indent} attempts to leave 
properly formatted \emph{boxed comments} unmodified.\\

\texttt{--comment-delimiters-on-blank-lines}
    
\noindent This option places the comment delimiters on blank lines.
Thus, a single line comment like \verb|/* Loving hug */| can be
transformed into:
\begin{verbatim}
/*
   Loving hug
 */
\end{verbatim}

\texttt{--star-left-side-of-comments} \\
\noindent This options causes stars to be placed at the beginning of multi-line
comments. Thus, the single-line comment above can be transformed into:
\begin{verbatim}
/*
 * Loving hug
 */
\end{verbatim}

\noindent \emph{indent} can also fix an existing comments style transforming 
the top comment to the bottom. 
\begin{verbatim}
/*A different kind of scent,
            for a different kind
    of comment.*/
 
/*
 * A different kind of scent,
 * for a different kind of comment.
 */
\end{verbatim}
 
Where indent is capable of handling the structure of comments, we require an 
additional component to handle the contents of the comment and enforce that a 
header comment is present.
\newpage
\subsection{Check Comments module}

The check comments module is a combination scanner parser.
The scanner ignores all characters until it matches the initial \texttt{/*}
indicating the start of a multi-line comment. The scanner then enters an 
exclusive start state in which it attempts to parse common comment labels which
would be present in either header or function comment blocks. The scanner then
passes off the tokens to the parser which verifies the syntax of the tokens and
prints to stdout in the case where the tokens do not parse correctly. upon 
matching \texttt{*/} while in the exclusive start state it will then return to
the scanners initial start state.

\scriptsize{\begin{verbatim}
    HEADER COMMENT                     FUNCTION COMMENT
 /*                                     /*
  * File:     A2P1.c                     * Name:        my_func
  * Author:   My Name 100123456          * Purpose:     ...
  * Date:     2011/09/12                 * Arguments:   ...
  * Version:  1.0                        * Output:      ...
  *                                      * Modifies:    ...
  * Purpose:                             * Returns:     ...
  * ...                                  * Assumptions: ...
  */                                     * Bugs:        ...
                                         * Notes:       ...
                                         */
\end{verbatim}}

\section{Indentation}

For code indentation the C style checker is required to check for correct 
indentation for many different levels of indentation. Unfortunately indent was 
not capable of this level of flexibility. Enter Vim's cindent to save the day.
Using Vim in ex mode we are able to pass the file in and indent it against a 
set of values specified in cinoptions. The set of values that the C style 
checker uses to adhere to the C style guidlines are:\\
\verb|        e0,n0,f0,{0,}0,:2,=2,l1,b0,t0,+4,c4,C1,(0,w1| \\
for a full listing or description of how these flags effect indentation see 
\autoref{chap:cinoptions}
\newpage
\subsection{Check Indent module}
After the file has been re-indented it is then compared against the original 
file using diff with a custom output format set.
\begin{verbatim}
diff \
    --old-line-format='%l
' \
    --new-line-format='%l
' \
    --old-group-format='%df%(f=l?:,%dl)d%dE
%<' \
    --new-group-format='%dea%dF%(F=L?:,%dL)
%>' \
    --changed-group-format='%df%(f=l?:,%dl)c%dF%(F=L?:,%dL)
%<---
%>$$$
' \
    --unchanged-group-format=""
\end{verbatim}

\noindent The output of the comparison is passed through a scanner program 
\texttt{indentR} which for each line that the two files differ it calculates 
the indentation level of the original file and the indentation level of the 
re-indented file, it then prints to stdout the line number that the difference 
occurred, or the range in the case where multiple lines are incorrectly 
indented, the code on that line, the current level of 
indentation, and the proper level of indentation.
\begin{verbatim}
   38:
       printf("do nothing\n");

   Code at indentation level [4] not at correct indentation [8].
\end{verbatim}



\section{Common Errors and Style}

In order to keep the composite checks scanner from becoming too complex a 
separate module for common errors was created to provide checks for some common
errors which are easily discovered with the use of a scanner program. The 
program checks conditional, loop and struct statements for white space after 
the token and left curly location on a separate line. It additional checks for 
tab and carriage return characters on a line. Once errors are identified it 
prints to stdout the line number that the error occurred, the code on that 
line, and the appropriate warning message.


\section{Composite Check}

The bulk of the complex checks are performed by the composite checks module 
which is a General LR parser. The code for the module is based on modified 
versions of the ANSI C lex specification, and the ANSI C yacc grammar 
description. \citep{ANSI}.

\subsection{Ignored statements}
The scanner specification has been modified to 
ignore preprocessor directives, include files, define statements, and comments.
This greatly simplifies the parsing of the files and reduces the number of 
errors the parser encounters. The scanner is capable of ignoring complex 
statements by entering an exclusive start state when it matches a particular
token.

\subsection{typedef statements}
When the scanner matches a typedef token the scanner enters an inclusive start
state, this means all tokens outside of the inclusive state are still matched,
however, tokens inside the inclusive state have priority. The inclusive state 
allows for typedef statements to consist of multiple identifier tokens without 
having to make another set of actions for those tokens. For example in the case
of \texttt{typedef int banana} on matching typedef it enters the 
\verb|<TYPEDEF>| inclusive start state, then upon matching the \texttt{int} 
token it performs the action associated with \texttt{int}. The same method of 
using inclusive start states exists for struct, and enum statements.

\subsection{Recurring variables}
When the scanner encounters variables that it will need to parse in the future 
like the name of a typedef, enumeration constant, or identifier name, it has to
store the variable name locally so that it will be able to look it up if it 
occurs again and return the appropriate token to the scanner. This is 
accomplished by maintaining a hash table of symbol values, when a variable name
is encountered the symbol table is checked for existence of the name and if it 
is not already in the table it is added.

\subsection{Grammar modifications and Extensions}
The ANSI C yacc grammar description has been modified to include extra 
production rules or in some cases small C routines, to initiate checks for 
style errors within a C file. 

\section{Format and white space}
The format and white space portion of the program is currently handled by 
indent. As in all parts related to indent, the file is passed through indent 
and then compared against the original file to display required changes to the 
user. Although indent is inflexible it does perform well in correcting errors
pertaining to white space and the format of the document.

\section{Web based interface}
The web based interface was a last minute upgrade to the project. It consists 
of three PHP files and one css stylesheet. The PHP files contain an uploader
script which allows the student to upload a file to the server, a PHP class 
file line\_sort.php which handles the sorting of the program output, and 
printing the output to the screen. Finally the stylecheck.php handles moving 
the files to a temporary location and passing the files as input to scruffy.sh
as well as handing the output of scruffy.sh to an instance of the line\_sort 
class. 
The web interface provides a clean output of the reported errors and a code 
view portion to the right which displays the users code in an editable view 
pane.
