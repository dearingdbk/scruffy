% Name           : scruffy.tex using theme -> hsrm-beamer-demo.sty
% Author         : Benjamin Weiss (benjamin.weiss@student.hs-rm.de)
% Version        : 0.3c
% Created on     : 05.05.2013
% Last Edited on : 16.05.2013
% Modified by    : Bruce Bearing 100036623
% Modified on    : 27.11.2013
% Copyright      : Copyright (c) 2013 by Benjamin Weiss. All rights reserved.
% License        : This file may be distributed and/or modified under the
%                  GNU Public License.
% Description    : Beamer presentation on scruffy using hsrm theme.

\documentclass[compress]{beamer}
%--------------------------------------------------------------------------
% Common packages
%--------------------------------------------------------------------------
\usepackage[T1]{fontenc}
\usepackage{graphicx}
\usepackage{pgfpages}
\usepackage{multicol}
\usepackage{tabularx,ragged2e}
\usepackage{booktabs}
\usepackage{listings}
\usepackage{hyperref}
\lstset{ %
language=[LaTeX]TeX,
basicstyle=\normalsize\ttfamily,
keywordstyle=,
numbers=left,
numberstyle=\tiny\ttfamily,
stepnumber=1,
showspaces=false,
showstringspaces=false,
showtabs=false,
breaklines=true,
frame=tb,
framerule=0.5pt,
tabsize=4,
framexleftmargin=0.5em,
framexrightmargin=0.5em,
xleftmargin=0.5em,
xrightmargin=0.5em
}

%--------------------------------------------------------------------------
% Load theme
%--------------------------------------------------------------------------
\usetheme{hsrm}

\usepackage{dtklogos} % must be loaded after theme
\usepackage{tikz}
\usetikzlibrary{mindmap,backgrounds}

%--------------------------------------------------------------------------
% General presentation settings
%--------------------------------------------------------------------------
\title{C Style Checker}
\subtitle{Flex and Bison Implementation.}
\date{\today}
\author{Bruce Dearing}
\institute{Acadia University}

%--------------------------------------------------------------------------
% Notes settings
%--------------------------------------------------------------------------
%\setbeameroption{show notes}
%\setbeameroption{show notes on second screen=left}

\begin{document}
%--------------------------------------------------------------------------
% Titlepage
%--------------------------------------------------------------------------

\maketitle

%\begin{frame}[plain]
%   \titlepage
%\end{frame}

%--------------------------------------------------------------------------
% Table of contents
%--------------------------------------------------------------------------
\section*{Overview}
\begin{frame}{Overview}
    % hideallsubsections - dont show the subsections of the slides.
    \tableofcontents[hideallsubsections]
\end{frame}

%--------------------------------------------------------------------------
% Content
%--------------------------------------------------------------------------

%--------------------------------------------------------------------------
% Introduction
%--------------------------------------------------------------------------
\section{Introduction}

%--------------------------------------------------------------------------
% Problem Description
%--------------------------------------------------------------------------
\subsection{Problem Description}

\begin{frame}{Problem Description}
        COMP 2103 students are required to format their programs according to a set of style guidelines.
        
\note
{
    Currently no style checker for C has been in use. \\ 
    The solution to this problem would require a program
    which \\ would maintain as much flexibility as possible while producing a detailed analysis of a students code. \\
    Ideally, the program would be used as the back end for a web page, \\
    so that someone could submit their program and the web server would return either confirmation that the style meets the 
    guidelines.\\
    or a list of non-conforming constructs.
}
\end{frame}

%--------------------------------------------------------------------------
% Solution Description
%--------------------------------------------------------------------------
\subsection{Solution Description}

\begin{frame}{Solution Description}
\begin{itemize}
    \item{Initial Approach}
        \begin{itemize}
            \item{Flex and Bison} 
        \end{itemize}
     
     \item{Second approach}
     \begin{itemize}
        \item{Flex and Bison}
        \item{GNU Indent}
     \end{itemize}

    \item{Third approach}
        \begin{itemize}
                \item{Flex and Bison}
        \item{GNU Indent}
        \item{Vim \texttt{'cindent'}}
        \end{itemize}
\end{itemize} 
\end{frame}


%--------------------------------------------------------------------------
% Solution Description Notes
%--------------------------------------------------------------------------
\note{
    \scriptsize{\begin{itemize}
    \item In order to provide a reasonable amount of functionality while still remaining flexible, 
        I initially chose to use a combination of LALR  compiler tools.
    \item Look-Ahead LR parser reading context free BNF grammar. \\
    \item Backus–Naur Form BNF 

     \item Unfortunately constructing a grammar that could also parse all the possible combinations of white space proved problematic.
     \item Therefore my second approach was to use a combination of compiler tools and a gnu open source project called indent.
     \item The indent program changes the appearance of a C program by inserting or deleting white space based on a set of supplied flags.
     \item indent can be used to format the code to the desired specifications.  

    \item As the name suggests gnu indent can handle indentation, however, it is not very flexible in implementing different levels of indentation.
    \item For indentation I use Vim's cindent, which is considerably more configurable using cinoptions.
    \end{itemize}}
}

%--------------------------------------------------------------------------
% Flex and bison
%--------------------------------------------------------------------------
\subsection{Flex and Bison}

\begin{frame}{What are Flex and Bison?}
    Flex and Bison are a set of tools originally designed for constructing compilers.
    They have proven to be very useful in building programs which handle
    structured input.
\end{frame}

%--------------------------------------------------------------------------
% Flex and Bison Notes
%--------------------------------------------------------------------------
\note
{
    \begin{itemize}
        \item Flex is a fast lexical analyser generator. 
        \item It is a tool for generating programs that perform pattern-matching on text.
        \end{itemize}
}


%--------------------------------------------------------------------------
% Flex Structure
%--------------------------------------------------------------------------
\begin{frame}[containsverbatim]{Flex structure}
\scriptsize{\begin{verbatim}
%{     /* Declarations and optiions */
int chars = 0;
int words = 0;
int lines = 0;
%}
%%    /* Patterns and actions.  */
[a-zA-Z]+ { words++; chars += strlen(yytext); }
\n        { chars++; lines++; }
.         { chars++; }
%%
 /* C code that is copied to the generated scanner. */
main(int argc, char **argv)
{
    yylex();
    printf("%8d%8d%8d\n", lines, words, chars);
}
\end{verbatim}}
\note{
    \begin{itemize}
        \item  The first section contains declarations and option settings. 
        \item The second section is a list of patterns and actions, and 
        \item the third section is C code that is copied to the generated scanner.
        \end{itemize}
     }
\end{frame} 


%--------------------------------------------------------------------------
% Bison Structure
%--------------------------------------------------------------------------
\begin{frame}[containsverbatim]{Bison Structure}
\scriptsize{\begin{verbatim}
%{                 /* declare options */
%}                  
%token ONE TWO EOL /* declare tokens */
%start start
%%               
start
    : exp EOL     /* grammar rules */
    ;
exp
    : ONE 
    | TWO
    ;
%%                /* C code that is copied to parser */
main(int argc, char **argv)
{
    yyparse();
}
yyerror(char *s)
{
    fprintf(stderr, "error: %s\n", s);
}
\end{verbatim}}
\note{
    \begin{itemize}
        \item Bison is a general-purpose parser generator that converts an 
            annotated context-free grammar into a deterministic LR or generalized LR (GLR) parser.
        \item The input file for the Bison utility is an annotated grammar file. 
            The general form of a Bison grammar file is as follows:
    \end{itemize}
}

\end{frame}

%--------------------------------------------------------------------------
% Bison Notes
%--------------------------------------------------------------------------

%--------------------------------------------------------------------------
% Implementation Description 
%--------------------------------------------------------------------------
\section{Implementation Description}


\begin{frame}{Style checker design}
    The C style checker is composed of four main components which perform the following
    checks:
    \begin{itemize}
        \item Comments;
        \item Indentation;
        \item Common errors, and Style;
        \item Format, and White space.
    \end{itemize}
        The components are tied together with a shell script which is accessed
    through a web based interface.
\end{frame}


%--------------------------------------------------------------------------
% Comments
%--------------------------------------------------------------------------
\subsection{Comments}
\begin{frame}[containsverbatim]{Comments}
The comment component of the style checker attempts to:
\begin{enumerate}
    \item verify that the program starts with a header comment similar to the following. and 
    \item that functions preceded by a short comment describing the function are correctly formatted.
    \end{enumerate}
\begin{multicols}{2}
\scriptsize{\begin{verbatim}
    HEADER COMMENT 
/*   
 * File:     A2P1.c
 * Author:   My Name 100123456  
 * Date:     2011/09/12  
 * Version:  1.0
 * 
 * Purpose: 
 * ...
 */ 
\end{verbatim}}
\scriptsize{\begin{verbatim}
FUNCTION COMMENT
 /*
  * Name:        my_func
  * Purpose:     ...
  * Arguments:   ...
  * Output:      ...
  * Modifies:    ...
  * Returns:     ...
  * Assumptions: ...
  * Bugs:        ...
  * Notes:       ...
  */
\end{verbatim}}
\end{multicols}
\end{frame}
%--------------------------------------------------------------------------
% Comments Design
%--------------------------------------------------------------------------
\begin{frame}[containsverbatim]{Comment Check Design}
The \textcolor{hsrmSec2}{\texttt{check\_comments}} program is a combined scanner parser.
\begin{verbatim}
%x COMMENT /* Exclusive start state.  */
%%
"/*"          { BEGIN(COMMENT); }
<COMMENT>"*/" { BEGIN(INITIAL); return(END_COMMENT);}

\end{verbatim}
\note{
The starting comment character '/*' triggers the scanner to enter an 'exclusive' start state \texttt{<COMMENT>} to capture
comment tokens and pass those tokens off to the parser.\\
The parser then handles the tokens and verifies the comment is complete.
}
\end{frame}

%--------------------------------------------------------------------------
% Comments Design NOTES 
%--------------------------------------------------------------------------



%--------------------------------------------------------------------------
% Comments Design
%--------------------------------------------------------------------------

\begin{frame}[containsverbatim]{Comment Check Design}
The parser is responsible for determining if the stream of tokens provided by the scanner conforms
to the grammar specification.
\scriptsize{\begin{verbatim}
%token  IDENTIFIER FILE_LBL AUTHOR START_COMMENT END_COMMENT VERSION DATE
%token  NAME ARGUMENTS OUTPUT MODIFIES RETURNS ASSUMPTIONS BUGS NOTES 
%token  PURPOSE LAST_VAL
%start program_body
%%
program_body
    : program_start program_comments
    ;
program_comments
    : comment_start
    | program_comments comment_start
    ;
program_start
    : START_COMMENT header_comment END_COMMENT {check_header();}
    ;
comment_start
    : START_COMMENT comment_body END_COMMENT
 ...
\end{verbatim}}
\end{frame}

%--------------------------------------------------------------------------
% Indentation
%--------------------------------------------------------------------------
\subsection{Indentation}
\begin{frame}[containsverbatim]{Indentation}
The indentation portion of the style checker attempts to verify the file has been indented correctly.
\begin{verbatim}
vim -e -s $temp_in < indent/vim_commands.scr
\end{verbatim}
\begin{exampleblock}{indent/vim\_commands.scr}
    \scriptsize{\begin{itemize}
        \item[:] \verb|set shiftwidth=4|
        \item[:] \verb|set cinoptions=e0,n0,f0,{0,}0,:2,=2,l1,b0,t0,+4,c4,C1,(0,w1|
        \item[:] \verb|normal gg=G|
        \item[:] \verb|wq|
    \end{itemize}}
\end{exampleblock}
\end{frame}

\note{

    The program begins by correctly indenting a temporary copy of the supplied file
    by passing the file into vim in execute mode and supplying a set of script commands.

    The temporary file is then compared against the original and the diff output
    is run through a lexical analyzer to report error messages
    when differences in indentation exist.  
}


%--------------------------------------------------------------------------
% Common Errors
%--------------------------------------------------------------------------
\subsection{Common Errors and Style}
\begin{frame}{Common Errors and Style}
The Common errors and style portion of the style checker consists of two components.
\begin{enumerate}
    \item  \textcolor{hsrmSec2}{\texttt{common\_errors}} program which initiates checks for:
        \begin{itemize}
            \item Common white space errors. and
            \item Code block bracket location.
        \end{itemize}
    \item  \textcolor{hsrmSec2}{\texttt{composite\_check}} program which combines the \\
            ANSI C grammar specification with a combination of additional 
            grammar productions to produce style error checks.  
\end{enumerate}

\end{frame}


%--------------------------------------------------------------------------
% Format and white space
%--------------------------------------------------------------------------
\subsection{Format and white space}
\begin{frame}{Format and white space}
The format portion of the style checker utilizes the GNU indent program to verify the file has been correctly formatted.
\note
{
    One problem that I ran into while using indent, is that it is not very flexible, there are a number of flags that can either be
    turned on or off, there is no middle ground if either way is acceptable.
    There are also some modifications that indent makes to code which cannot be turned off. No flag exists to change its implementation.
    Due to this lack of flexibility, there are some style errors that it picks up that are acceptable, and a certain amount of redundancy
    in reporting.
}
\end{frame}


%--------------------------------------------------------------------------
% Possible Extensions
%--------------------------------------------------------------------------
\section{Possible Extensions}
    \note
    {
        I made an initial attempt to implement Preprocessing the files before checking for errors. 
        Running through the C preprocessor proved very useful in eliminating portions of the file which are difficult to parse without it.
        It can be implemented to replaces macro definitions and join broken strings and variable together.
        and line position can be easily recalculated with the linemarkers the preprocessor inserts into the outfile.
        where I ran into problems with implementation is when the C file included local header files. 
        In order to implement the preprocessor portion a multiple file upload portion would be required.
        This would also eliminate the need for a GLR parser in the Common Errors and Style portion.
    }
\begin{frame}{Possible Extensions}
\begin{description}
    \item[C Preprocessor] \hfill \\
        Construct the C style checker to first run the files through the C preprocessor.

    \item[Continue Productions] \hfill \\
        Continue constructing grammar productions to produce checks for additional style errors.

    \item[Eliminate or extend indent] \hfill \\
        Extend or modify indent's source code to enhance its flexibility.
    \note
    {
        I initially made a list of all the checks that checkstyle for java was able to perform.\\
        After removing the java specific items and any which conflicted with the style guidelines \\
        I was left with a baseline list of 39 checks. \\
        I was to able code checks for 28 of them.\\
        
        \texttt{the source code for indent 2.2.10 could be modified and combined into a future version of this 
        program which could eliminate some of the redundancy caused by indents lack of flexibility.} 
    
    }
\end{description}

    
\end{frame}

%--------------------------------------------------------------------------
% Summary and Conclusions
%--------------------------------------------------------------------------
\section{Summary and Conclusions}
\begin{frame}{Available Checks}
    The C style checker provides the following checks.
    \begin{multicols}{2}
    \scriptsize{\begin{enumerate}
        \item Array Trailing Comma
        \item Type Name
        \item Default Comes Last
        \item Empty Block
        \item Empty Statement 
        \item Fall Through
        \item File Length
        \item Indentation
        \item Left Curly
        \item Line Length
        \item Local Variable Name
        \item Magic Number
        \item Method Name
        \item Method Param Pad 
        \item Missing Switch Default
        \item Modified Control Variable
        \item One Statement Per Line
        \item Parameter Name
        \item Paren Pad
        \item Right Curly
        \item Type Name
        \item Type cast Paren Pad
        \item Header comment
        \item White space After
        \item White space Around
        \item Multiple variable declarations with initializations
        
    \note{
    
    \tiny{\begin{itemize}
        \item Array Trailing Comma - Checks if array initialization contains optional trailing comma.
        \item Type Name - Checks that type names conform to a format specified by the style guide. 
        \item Default Comes Last - Check that the \texttt{default} case is after all the cases in a switch statement.
         \item Empty Block - Checks for empty code blocks.
        \item Empty Statement - Detects empty statements (standalone ';'). 
        \item Fall Through - Checks for fall through in \texttt{switch} statements Finds locations where a case contains code 
        but lacks a \texttt{break}, \texttt{return}, or \texttt{continue} statement.
        \item File Length - Checks for long source files.
        \item Indentation - Checks correct indentation of C Code.
        \item Left Curly - Checks the placement of left curly braces on types, functions and other blocks. 
        \item Line Length - Checks for long lines.
        \item Local Variable Name - Checks that local, variable names conform to a format specified by the style guide.
        \item Magic Number - Checks for magic numbers.
        \item Method Name - Checks that method names conform to a format specified by the style guide.
        \item Method Param Pad - Checks the padding between the identifier of a method definition and the 
        left parenthesis of the parameter list.
        \item Missing Switch Default - Checks that switch statement has \texttt{default} case.
    \end{itemize}}}
    \note{
    \tiny{\begin{itemize}
        \item One Statement Per Line - Checks there is only one statement per line.
        \item Parameter Name - Checks that parameter names conform to a format specified by the format property.
        \item Paren Pad - Checks the padding of parentheses; that is whether a space is required after a 
        left parenthesis and before a right parenthesis.
        \item Right Curly - Checks the placement of right curly braces.
        \item Type Name - Checks that type names conform to a format specified by the style guide.
        \item Type cast Paren Pad - Checks the padding of parentheses for typecasts.
        \item Header comment - Detects files missing header comments.
        \item White space After - Checks that a token is followed by whitespace.
        \item White space Around Checks that a token is surrounded by whitespace.
        \item Multiple variable declarations with initializations - Checks that an initialization of a variable is on its own line.
    \end{itemize}}
     }

\end{enumerate}}
\end{multicols}
\end{frame}

\subsection{Required Software}
    \begin{frame}
    Required Software:
        \begin{itemize}
            \item VIM - Vi IMproved - version 7.3.547
            \item flex 2.5.35
            \item bison (GNU Bison) 2.7.12-4996 (any version above 2.5).
            \item GNU indent 2.2.11
            \item gcc 4.7.2
        \end{itemize}
    \end{frame}

\subsection{Demonstration}
\begin{frame}{Demonstration}
\href{http://localhost/uploader.php}{Demonstration}
\end{frame}

\section{Questions?}


\end{document}






